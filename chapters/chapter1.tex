\chapter{有限状态自动机}

\section{字母表}

\subsection{字母表(Alphabet)}

字母表是一个非空的有限集合,一般用$ \Sigma $表示,集合中的元素被称为符号/字符(symbol)。\\

例如:

\begin{itemize}
	\item $ \Sigma = \{0, 1\} $:二进制数的集合。
	\item $ \Sigma = \{a, b, \cdots, z\} $:小写字母集合。
	\item $ \Sigma = \{(, ), [, ], \{, \}\} $:括号集合。
\end{itemize}

\vspace{0.5cm}

\subsection{串(String)}

串是一个由字母表中的字符组成的有限序列。\\

例如:

\begin{itemize}
	\item 0011和11是$ \Sigma = \{0, 1\} $上的串。
	\item abc和bbb是$ \Sigma = \{a, b, \cdots, z\} $上的串。
	\item (())和(()是$ \Sigma = \{(, ), [, ], \{, \}\} $上的串。
\end{itemize}

\vspace{0.5cm}

\subsubsection{空串}

空串使用$ \epsilon $表示。\\

\subsubsection{串的长度}

\begin{itemize}
	\item $ |0010| = 4 $
	\item $ |aa| = 2 $
	\item $ |\epsilon| = 0 $
\end{itemize}

\vspace{0.5cm}

\subsubsection{前缀(prefix)}

\begin{itemize}
	\item aa是aaabc的前缀
	\item aaab是aaabc的前缀
	\item aaabc是aaabc的前缀
\end{itemize}

\vspace{0.5cm}

\subsubsection{后缀(suffix)}

\begin{itemize}
	\item bc是aaabc的后缀
	\item abc是aaabc的后缀
	\item aaabc是aaabc的后缀
\end{itemize}

\vspace{0.5cm}

\subsubsection{子串(substring)}

\begin{itemize}
	\item ab是aaabc的子串
	\item aaa是aaabc的子串
	\item aaabc是aaabc的子串
\end{itemize}

\vspace{0.5cm}

\subsubsection{连接(concatenation)}

当$ \omega = abd $,$ \alpha = ce $,那么$ \omega\alpha = abdce $。\\

\subsubsection{指数(exponentiation)}

当$ \omega = abd $,那么$ \omega^3 = abdabdabd $,$ \omega^0 = \epsilon $。\\

\subsubsection{反转(reversal)}

当$ \omega = abd $,那么$ \omega^R = dba $。\\

\subsection{克林闭包(Kleene Closure)}

$ \Sigma^k $用于表示所有在字母表$ \Sigma $上的长度为$ k $的串的集合。\\

例如,$ \Sigma = \{a, b\} $,那么$ \Sigma^2 = \{ab, ba, aa, bb\} $,$ \Sigma^0 = \{\epsilon\} $。\\

克林闭包$ \Sigma^* $用于表示所有在字母表$ \Sigma $上能够组成的串的集合。\\

\vspace{-1cm}

\begin{align}
	\Sigma^* = \Sigma^0 \cup \Sigma^1 \cup \Sigma^2 \cup \cdots = \bigcup_{k \ge 0}\Sigma^k
\end{align}

正闭包$ \Sigma^+ $则是在$ \Sigma^* $中除了空串以外的所有串的集合。\\

\vspace{-1cm}

\begin{align}
	\Sigma^+ = \Sigma^1 \cup \Sigma^2 \cup \Sigma^3 \cup \cdots = \bigcup_{k > 0}\Sigma^k
\end{align}

\newpage

\section{语言}

\subsection{语言(Language)}

语言是一个字母表中所构成串的集合。\\

例如,$ \Sigma = \{a, b, c, \cdots, z\} $,那么所有英语单词所构成的集合$ L $就是字母表$ \Sigma $上的语言。\\

假设$ A = \{good, bad\} $和$ B = \{boy, girl\} $是两个语言,语言之间可以进行以下操作。\\

\subsubsection{并集(union)}

\vspace{-1cm}

\begin{align}
	A \cup B = \{x\ |\ x \in A \text{ or } x \in B\}
\end{align}

$ A \cup B = \{good, bad, boy, girl\} $\\

\subsubsection{连接(concatenation)}

\vspace{-1cm}

\begin{align}
	A \circ B = \{xy\ |\ x \in A \text{ or } y \in B\}
\end{align}

$ A \circ B = \{goodboy, goodgirl, badboy, badgirl\} $\\

\subsubsection{闭包}

\vspace{-1cm}

\begin{align}
	A^* = \{x_1, x_2, \cdots, x_k\ |\ k \ge 0 \text{ and each } x_i \in A\}
\end{align}

$ A^* = \{\epsilon, good, bad, goodgood, goodbad, badgood, badbad, goodgoodgood, goodgoodbad, \cdots\} $\\

语法和语言与自动机理论密切相关,它们是许多软件实现的基础,例如编译器/解释器、文本编辑器、文本搜索、系统验证等。\\

在自动机理论中,要处理的问题就是判断一个给定的串是否属于某个语言。\\

例如:

\begin{itemize}
	\item $ 0^*10^* $: 只包含一个1的串的集合。
	
	\item $ \Sigma^*1\Sigma^* $:至少有一个1的串的集合。
	
	\item $ \Sigma^*001\Sigma^* $:包含子串001的串的集合。
	
	\item $ (\Sigma\Sigma)^* $:长度为偶数的串的集合。
	
	\item $ (\Sigma\Sigma\Sigma)^* $:长度为3的倍数的串的集合。
\end{itemize}

\newpage

\section{确定性有限状态自动机}

\subsection{确定性有限状态自动机(DFA, Deterministic Finite Automatons)}

有限状态机(FSM, Finite State Machine)用于决定程序当前状态和状态间的切换,状态机最终只能指向一个结果。\\

\begin{figure}[H]
	\centering
	\begin{tikzpicture}[node distance=4cm]
		\node[state, initial, draw, align=center] (s1) {off};
		\node[state, right of=s1, draw, align=center] (s2) {on};

		\draw[->] (s1) edge[above, bend left] node{1} (s2);
		\draw[->] (s2) edge[above, bend left] node{0} (s1);
	\end{tikzpicture}
	\caption{有限状态机}
\end{figure}

DFA使用一个五元组$ (Q, \Sigma, \delta, q_0, F) $表示,其中

\begin{itemize}
	\item $ Q $:状态的集合
	\item $ \Sigma $:字母表
	\item $ \delta $:状态转移函数(transition function) 
	\item $ q_0 $:初始状态
	\item $ F $:终结状态集合
\end{itemize}

\newpage

\begin{figure}[H]
	\centering
	\begin{tikzpicture}[node distance=5cm]
		\node[state, initial, draw, align=center] (s1) {wait for\\call 0};
		\node[state, right of=s1, draw, align=center] (s2) {wait for\\ACK/NAK 0};
		\node[state, below of=s2, draw, align=center] (s3) {wait for\\call 1};
		\node[state, left of=s3, draw, align=center] (s4) {wait for\\ACK/NAK 1};

		\draw[->] (s1) edge[above, bend left] node{} (s2);
		\draw[->] (s2) edge[loop right] node{} (s2);
		\draw[->] (s2) edge[above, bend left] node{} (s3);
		\draw[->] (s3) edge[above, bend left] node{} (s4);
		\draw[->] (s4) edge[loop left] node{} (s4);
		\draw[->] (s4) edge[above, bend left] node{} (s1);

		\draw (3,3) node {rdt\_send(data)};
		\draw (0,2.75) -- (6,2.75);
		\draw (3,2.5) node {sndpkt = make\_pkt(0, data)};
		\draw (2,2) node {udt\_send(sndpkt)};

		\draw (8.5,2.5) node {rdt\_rcv(rcvpkt) \&\&};
		\draw (8.3,2) node {(corrupt(rcvpkt) ||};
		\draw (8.1,1.5) node {is\_nak(rcvpkt))};
		\draw (6.5,1.25) -- (10.5,1.25);
		\draw (8.5,1) node {udt\_send(sndpkt)};

		\draw (8,-2) node {rdt\_rcv(rcvpkt) \&\&};
		\draw (8,-2.5) node {!corrupt(rcvpkt) \&\&};
		\draw (7.6,-3) node {is\_ack(rcvpkt))};
		\draw (6,-3.25) -- (10,-3.25);
		\draw (7.5,-3.5) node {$ \wedge $};

		\draw (4,-7) node {rdt\_send(data)};
		\draw (1,-7.25) -- (7,-7.25);
		\draw (4,-7.5) node {sndpkt = make\_pkt(1, data)};
		\draw (3,-8) node {udt\_send(sndpkt)};

		\draw (-2,-7) node {rdt\_rcv(rcvpkt) \&\&};
		\draw (-2.2,-7.5) node {(corrupt(rcvpkt) ||};
		\draw (-2.4,-8) node {is\_nak(rcvpkt))};
		\draw (-4,-8.25) -- (0,-8.25);
		\draw (-2,-8.5) node {udt\_send(sndpkt)};

		\draw (-3,-2) node {rdt\_rcv(rcvpkt) \&\&};
		\draw (-3,-2.5) node {!corrupt(rcvpkt) \&\&};
		\draw (-3.4,-3) node {is\_ack(rcvpkt))};
		\draw (-5,-3.25) -- (-1,-3.25);
		\draw (-3,-3.5) node {$ \wedge $};
	\end{tikzpicture}
	\caption{rdt 2.1发送端}
\end{figure}

\newpage