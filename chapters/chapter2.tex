\chapter{上下文无关语言}

\section{上下文无关文法}

\subsection{上下文无关文法(CFG, Context Free Grammar)}

CFG能够描述某些具有递归结构的特征,它有足够强的语言表达力来表示大多数编程语言的语法。\\

CFG由一个四元组$ (V, T, P, S) $表示:

\begin{itemize}
    \item $ V $:变元(variable)/非终结符(non-terminal)集合,用大写字母表示。
    \item $ T $:终结符(terminal)集合,用小写字母表示。
    \item $ P $:产生式(production)集合。
    \item $ S $:开始符号。
\end{itemize}

\vspace{0.5cm}

一个文法由一组替换规则产生。产生式集合

\vspace{-1cm}

\begin{align*}
    A & \rightarrow \alpha_1 \\
    A & \rightarrow \alpha_2 \\
      & \cdots               \\
    A & \rightarrow \alpha_k
\end{align*}

可以被写成$ A \rightarrow \alpha_1\ |\ \alpha_2\ |\ \cdots\ |\ \alpha_k\ $的形式。\\

\subsection{推导(Derivation)}

推导用于确定符合文法规则的串的集合,即用来确定一个语言。\\

推导从开始符号开始,通过产生式进行替换,得到最终结果。\\

例如$ E \rightarrow E + E\ |\ E * E\ |\ (E)\ |\ id $,由开始符号$ E $可以推导出$ (id + id) * id $。

\vspace{-1cm}

\begin{align*}
    E & \Rightarrow E * E          \\
      & \Rightarrow (E) * E        \\
      & \Rightarrow (E) * id       \\
      & \Rightarrow (E + E) * id   \\
      & \Rightarrow (E + id) * id  \\
      & \Rightarrow (id + id) * id
\end{align*}

解析树(parse tree)是描述推导的一种直观方法。\\

\begin{figure}[H]
    \centering
    \begin{tikzpicture}[
            -,
            level distance=1.5cm,
            level 1/.style={sibling distance=4cm},
            level 2/.style={sibling distance=3cm},
            level 3/.style={sibling distance=2cm},
            level 4/.style={sibling distance=1cm}
        ]
        \node {$ E $}
        child {
                node {$ E $}
                child {
                        node {$ ( $}
                    }
                child {
                        node {$ E $}
                        child {
                                node {$ E $}
                                child {
                                        node {$ id $}
                                    }
                            }
                        child {
                                node {$ + $}
                            }
                        child {
                                node {$ E $}
                                child {
                                        node {$ id $}
                                    }
                            }
                    }
                child {
                        node {$ ) $}
                    }
            }
        child {
                node {$ * $}
            }
        child {
                node {$ E $}
                child {
                        node {$ id $}
                    }
            };
    \end{tikzpicture}
    \caption{分析树}
\end{figure}

\vspace{0.5cm}

如果只关注语义分析和代码生成所需的信息,可以将分析树简化为一棵抽象语法树(abstract syntex tree)。\\

\begin{figure}[H]
    \centering
    \begin{tikzpicture}[
            -,
            level distance=1.5cm,
            level 1/.style={sibling distance=3cm},
            level 2/.style={sibling distance=2cm},
            level 3/.style={sibling distance=2cm}
        ]
        \node {$ * $}
        child {
                node {$ + $}
                child {
                        node {$ id $}
                    }
                child {
                        node {$ id $}
                    }
            }
        child {
                node {$ id $}
            };
    \end{tikzpicture}
    \caption{抽象语法树}
\end{figure}

\vspace{0.5cm}

\subsection{二义性(Ambiguity)}

在推导的过程中涉及到同级别表达式的替换,因此按顺序可以分为最左推导(leftmost derivation)和最右推导(rightmost derivation)。\\

文法的二义性,是指对于符合文法规则的同一个句子,存在两种可能的分析树。\\

例如$ E \rightarrow E + E\ |\ E * E\ |\ (E)\ |\ x\ |\ y\ |\ z $,使用最左推导会对$ x + y * z $产生两个不同的分析树。\\

\begin{figure}[H]
    \centering
    \begin{tikzpicture}[
            -,
            level distance=1.5cm,
            level 1/.style={sibling distance=3cm},
            level 2/.style={sibling distance=2cm},
            level 3/.style={sibling distance=2cm}
        ]
        \node {$ E $}
        child {
                node {$ E $}
                child {
                        node {$ x $}
                    }
            }
        child {
                node {$ + $}
            }
        child {
                node {$ E $}
                child {
                        node {$ y $}
                    }
                child {
                        node {$ * $}
                    }
                child {
                        node {$ z $}
                    }
            };
    \end{tikzpicture}
\end{figure}

\begin{figure}[H]
    \centering
    \begin{tikzpicture}[
            -,
            level distance=1.5cm,
            level 1/.style={sibling distance=3cm},
            level 2/.style={sibling distance=2cm},
            level 3/.style={sibling distance=2cm}
        ]
        \node {$ E $}
        child {
                node {$ E $}
                child {
                        node {$ x $}
                    }
                child {
                        node {$ + $}
                    }
                child {
                        node {$ y $}
                    }
            }
        child {
                node {$ * $}
            }
        child {
                node {$ E $}
                child {
                        node {$ z $}
                    }
            };
    \end{tikzpicture}
\end{figure}

\vspace{0.5cm}

产生二义性的原因在于运算符之间的优先级在文法中并没有体现。消除二义性的办法就是在文法中引入一个中间量。

\vspace{-1cm}

\begin{align*}
    E & \rightarrow E + T\ |\ T \\
    T & \rightarrow T * F\ |\ F \\
    F & \rightarrow id\ |\ (E)
\end{align*}

\begin{figure}[H]
    \centering
    \begin{tikzpicture}[
            -,
            level distance=1.5cm,
            level 1/.style={sibling distance=3cm},
            level 2/.style={sibling distance=2cm},
            level 3/.style={sibling distance=2cm}
        ]
        \node {$ E $}
        child {
                node {$ E $}
                child {
                        node {$ T $}
                        child {
                                node {$ F $}
                                child {
                                        node {$ x $}
                                    }
                            }
                    }
            }
        child {
                node {$ + $}
            }
        child {
                node {$ T $}
                child {
                        node {$ T $}
                        child {
                                node {$ F $}
                                child {
                                        node {$ y $}
                                    }
                            }
                    }
                child {
                        node {$ * $}
                    }
                child {
                        node {$ F $}
                        child {
                                node {$ z $}
                            }
                    }
            };
    \end{tikzpicture}
\end{figure}

\newpage

\section{CNF}

\subsection{上下文无关语言}

CFG可以用来表示语言$ \{a^nb^n\ |\ n \ge 0\} $:

\vspace{-1cm}

\begin{align*}
    S & \rightarrow aSb \\
    S & \rightarrow ab
\end{align*}

例如根据$ S $可以生成生成aaabbb:

\vspace{-1cm}

\begin{align*}
    S & \Rightarrow aSb    \\
      & \Rightarrow aaSbb  \\
      & \Rightarrow aaabbb
\end{align*}

CFG好还可以用于表示$ a $和$ b $出现相等次数的语言,例如babaab:

\vspace{-1cm}

\begin{align*}
    S & \rightarrow aB\ |\ bA        \\
    A & \rightarrow a\ |\ aS\ |\ bAA \\
    B & \rightarrow b\ |\ bS\ |\ aBB
\end{align*}

设计CFG需要一定的创造力,大部分复杂的CFG可以由多个简单的CFG并集组成。\\

例如设计一个能够表示语言$ \{0^n1^n\ |\ n \ge 0\} \cup \{1^n0^n\ |\ n \ge 0\} $的CFG。\\

这两个部分可以分别表示为:

\vspace{-1cm}

\begin{align*}
    S_1 & \rightarrow 0S_{1}1\ |\ \epsilon \\
    S_2 & \rightarrow 1S_{2}0\ |\ \epsilon
\end{align*}

只需合并这两个部分,即可得到最终的CFG:

\vspace{-1cm}

\begin{align*}
    S   & \rightarrow S_1\ |\ S_2          \\
    S_1 & \rightarrow 0S_{1}1\ |\ \epsilon \\
    S_2 & \rightarrow 1S_{2}0\ |\ \epsilon
\end{align*}

\vspace{0.5cm}

\subsection{乔姆斯基范式(CNF, Chomsky Normal Form)}

CNF在保留相同语言的同时对语法规则施加了一些限制,好处是可以避免解析过程中的歧义问题,另一个好处就是为解析的复杂度提供了一个上限。\\

CNF规定每条CFG的每一条规则都必须满足:

\begin{enumerate}
    \item $ S \rightarrow \epsilon $:开始变元$ S $可以为空。
    \item $ A \rightarrow BC $:单个变元可以推导出两个变元,其中$ B $、$ C $不能为开始变元。
    \item $ A \rightarrow a $:单个变元可以被终结符替换。
    \item 不能出现单个变元推导出单个变元。
\end{enumerate}

\vspace{0.5cm}

将CFG转换为CNF的步骤为:

\begin{enumerate}
    \item 添加新的开始变元:确保开始变元始终在规则的左侧。
    \item 消除所有$ \epsilon $规则:消除从变元到空字符的规则。
    \item 消除所有$ A \rightarrow B $规则:消除单个变元到单个变元的规则。
    \item 添加变元:为了满足$ A \rightarrow BC $的规则,需要将$ A \rightarrow BCD $替换为$ A \rightarrow ED $,即添加变元$ E \rightarrow BC $。
\end{enumerate}

\vspace{0.5cm}

例如将CFG转换为CNF:

\vspace{-1cm}

\begin{align*}
    S & \rightarrow ABA             \\
    A & \rightarrow aA\ |\ \epsilon \\
    B & \rightarrow bB\ |\ \epsilon
\end{align*}

\subsubsection{消除所有$ \epsilon $规则}

将$ A \rightarrow \epsilon $的规则,替换到出现$ A $的规则中:

\vspace{-1cm}

\begin{align*}
    S & \rightarrow ABA\ |\ \textcolor{red}{BA\ |\ AB\ |\ B} \\
    A & \rightarrow aA\ |\ \textcolor{red}{a}                \\
    B & \rightarrow bB\ |\ \epsilon
\end{align*}

将$ B \rightarrow \epsilon $的规则,替换到出现$ B $的规则中:

\vspace{-1cm}

\begin{align*}
    S & \rightarrow ABA\ |\ BA\ |\ AB\ |\ B\ |\ \textcolor{red}{AA\ |\ A} \\
    A & \rightarrow aA\ |\ a                                              \\
    B & \rightarrow bB\ |\ \textcolor{red}{b}
\end{align*}

\subsubsection{消除所有$ A \rightarrow B $规则}

在$ S $中出现了单个变元到单个变元的情况,将这些规则进一步替换:

\vspace{-1cm}

\begin{align*}
    S & \rightarrow ABA\ |\ BA\ |\ AB\ |\ \textcolor{red}{bB\ |\ b}\ |\ AA\ |\ \textcolor{red}{aA\ |\ a} \\
    A & \rightarrow aA\ |\ a                                                                             \\
    B & \rightarrow bB\ |\ b
\end{align*}

目前,$ S \rightarrow BA $、$ S \rightarrow AA $、$ S \rightarrow AB $、$ S \rightarrow a $、$ S \rightarrow b $、$ A \rightarrow a $、$ B \rightarrow b $这些规则已经满足了CNF的要求:

\vspace{-1cm}

\begin{align*}
    S & \rightarrow ABA\ |\ \textcolor{ForestGreen}{BA\ |\ AB}\ |\ bB\ |\ \textcolor{ForestGreen}{b\ |\ AA}\ |\ aA\ |\ \textcolor{ForestGreen}{a} \\
    A & \rightarrow aA\ |\ \textcolor{ForestGreen}{a}                                                                                             \\
    B & \rightarrow bB\ |\ \textcolor{ForestGreen}{b}
\end{align*}

\subsubsection{添加变元}

为了消除$ A \rightarrow BCD $这种情况,需要添加新的变元进行替换。\\

假设$ X \rightarrow AB $:

\vspace{-1cm}

\begin{align*}
    S                  & \rightarrow \textcolor{red}{XA}\ |\ \textcolor{ForestGreen}{BA\ |\ AB}\ |\ bB\ |\ \textcolor{ForestGreen}{b\ |\ AA}\ |\ aA\ |\ \textcolor{ForestGreen}{a} \\
    A                  & \rightarrow aA\ |\ \textcolor{ForestGreen}{a}                                                                                                             \\
    B                  & \rightarrow bB\ |\ \textcolor{ForestGreen}{b}                                                                                                             \\
    \textcolor{red}{X} & \textcolor{red}{\rightarrow} \textcolor{red}{AB}
\end{align*}

同时为了满足CNF规则中$ A \rightarrow BC $的要求,需要对如$ A \rightarrow aA $这样的规则进行替换。\\

假设$ A_1 \rightarrow a $、$ B_1 \rightarrow b $:

\vspace{-1cm}

\begin{align*}
    S                    & \rightarrow \textcolor{ForestGreen}{XA\ |\ BA\ |\ AB}\ |\ \textcolor{red}{B_1B}\ |\ \textcolor{ForestGreen}{b\ |\ AA}\ |\ \textcolor{red}{A_1A}\ |\ \textcolor{ForestGreen}{a} \\
    A                    & \rightarrow \textcolor{red}{A_1A}\ |\ \textcolor{ForestGreen}{a}                                                                                                               \\
    B                    & \rightarrow \textcolor{red}{B_1B}\ |\ \textcolor{ForestGreen}{b}                                                                                                               \\
    X                    & \rightarrow \textcolor{ForestGreen}{AB}                                                                                                                                        \\
    \textcolor{red}{A_1} & \textcolor{red}{\rightarrow} \textcolor{red}{a}                                                                                                                                \\
    \textcolor{red}{B_1} & \textcolor{red}{\rightarrow} \textcolor{red}{b}
\end{align*}

这样就完成了CFG到CNF的转换,语法中的每条规则都满足了CNF的要求。

\newpage

\section{PDA}

\subsection{下推自动机(PDA, Pushdown Automata)}

DFA和NFA由于受限于存储空间的问题,不能识别类似于$ \{a^nb^n\ |\ n \ge 0\} $这种语言。PDA通过一个栈(stack)解决了这个问题。PDA与CFG的功能的等价的。\\

\begin{figure}[H]
    \centering
    \begin{tikzpicture}[node distance=0mm, every node/.style={minimum size=8mm}]
        \node[draw] (deb) {State Control};

        { [start chain=1]
        \node[draw] [on chain] at (3,-2){$ a $};
        \node[draw] [on chain] {$ a $};
        \node[draw] [on chain] {$ b $};
        \node[draw] [on chain] {$ b $};
        }
        \node at (4.5,-3) {input};

        { [start chain=2 going below]
        \node[draw] [on chain] at (-2,-3){0};
        \node[draw] [on chain] {$ x $};
        \node[draw] [on chain] {$ y $};
        \node[draw] [on chain] {$ z $};
        }
        \node at (-2,-6.5) {stack};

        \begin{scope}[->,>=latex']
            \draw (deb.east) -| (1-1);
            \draw (deb.south) |- (2-1.east);
        \end{scope}
    \end{tikzpicture}
    \caption{PDA}
\end{figure}

PDA由一个六元组$ (Q, \Sigma, \Gamma, \delta, q_0, F) $表示:

\begin{itemize}
    \item $ Q $:状态集合
    \item $ \Sigma $:输入字母表
    \item $ \Gamma $:栈字母表
    \item $ \delta $:状态转移函数
    \item $ q_0 $:初始状态
    \item $ F $:终结状态集合
\end{itemize}

\vspace{0.5cm}

例如状态转移函数$ \delta(q_1, a, b) = \{(q_2, \epsilon)\} $表示,在状态$ q_1 $时,如果输入字符为$ a $,并且栈顶元素为$ b $,那么就将$ a $消耗掉,并将$ b $出栈,进入状态$ q_2 $。在PDA中可表示为$ a, b \rightarrow \epsilon $。\\

例如状态转移函数$ \delta(q_3, \epsilon, b) = \{(q_4, a), (q_5, b)\} $表示,在状态$ q_3 $时,如果输入字符为空,并且栈顶元素为$ b $,那么有两种选择:

\begin{enumerate}
    \item 使用$ a $代替栈顶元素$ b $,并进入状态$ q_4 $。
    \item 栈保持原样($ b $为栈顶),并进入状态$ q_5 $。
\end{enumerate}

