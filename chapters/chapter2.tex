\chapter{上下文无关语言}

\section{上下文无关文法}

\subsection{上下文无关文法(CFG, Context Free Grammar)}

CFG能够描述某些具有递归结构的特征,它有足够强的语言表达力来表示大多数编程语言的语法。\\

CFG由一个四元组$ (V, T, P, S) $表示:

\begin{itemize}
    \item $ V $:变元(variable)/非终结符(non-terminal)集合,用大写字母表示。
    \item $ T $:终结符(terminal)集合,用小写字母表示。
    \item $ P $:产生式(production)集合。
    \item $ S $:开始符号。
\end{itemize}

一个文法由一组替换规则产生。产生式集合

\vspace{-1cm}

\begin{align*}
    A & \rightarrow \alpha_1 \\
    A & \rightarrow \alpha_2 \\
      & \cdots               \\
    A & \rightarrow \alpha_k
\end{align*}

可以被写成$ A \rightarrow \alpha_1\ |\ \alpha_2\ |\ \cdots\ |\ \alpha_k\ $的形式。\\

\subsection{推导(Derivation)}

推导用于确定符合文法规则的串的集合,即用来确定一个语言。\\

推导从开始符号开始,通过产生式进行替换,得到最终结果。\\

例如$ E \rightarrow E + E\ |\ E * E\ |\ (E)\ |\ id $,由开始符号$ E $可以推导出$ (id + id) * id $。

\vspace{-1cm}

\begin{align*}
    E & \Rightarrow E * E          \\
      & \Rightarrow (E) * E        \\
      & \Rightarrow (E) * id       \\
      & \Rightarrow (E + E) * id   \\
      & \Rightarrow (E + id) * id  \\
      & \Rightarrow (id + id) * id
\end{align*}

解析树(parse tree)是描述推导的一种直观方法。\\

\begin{figure}[H]
    \centering
    \begin{tikzpicture}[
            -,
            level distance=1.5cm,
            level 1/.style={sibling distance=4cm},
            level 2/.style={sibling distance=3cm},
            level 3/.style={sibling distance=2cm},
            level 4/.style={sibling distance=1cm}
        ]
        \node {$ E $}
        child {
                node {$ E $}
                child {
                        node {$ ( $}
                    }
                child {
                        node {$ E $}
                        child {
                                node {$ E $}
                                child {
                                        node {$ id $}
                                    }
                            }
                        child {
                                node {$ + $}
                            }
                        child {
                                node {$ E $}
                                child {
                                        node {$ id $}
                                    }
                            }
                    }
                child {
                        node {$ ) $}
                    }
            }
        child {
                node {$ * $}
            }
        child {
                node {$ E $}
                child {
                        node {$ id $}
                    }
            };
    \end{tikzpicture}
    \caption{分析树}
\end{figure}

\vspace{0.5cm}

如果只关注语义分析和代码生成所需的信息,可以将分析树简化为一棵抽象语法树(abstract syntex tree)。\\

\begin{figure}[H]
    \centering
    \begin{tikzpicture}[
            -,
            level distance=1.5cm,
            level 1/.style={sibling distance=3cm},
            level 2/.style={sibling distance=2cm},
            level 3/.style={sibling distance=2cm}
        ]
        \node {$ * $}
        child {
                node {$ + $}
                child {
                        node {$ id $}
                    }
                child {
                        node {$ id $}
                    }
            }
        child {
                node {$ id $}
            };
    \end{tikzpicture}
    \caption{抽象语法树}
\end{figure}

\vspace{0.5cm}

\subsection{二义性(Ambiguity)}

在推导的过程中涉及到同级别表达式的替换,因此按顺序可以分为最左推导(leftmost derivation)和最右推导(rightmost derivation)。\\

文法的二义性,是指对于符合文法规则的同一个句子,存在两种可能的分析树。\\

例如$ E \rightarrow E + E\ |\ E * E\ |\ (E)\ |\ x\ |\ y\ |\ z $,使用最左推导会对$ x + y * z $产生两个不同的分析树。\\

\begin{figure}[H]
    \centering
    \begin{tikzpicture}[
            -,
            level distance=1.5cm,
            level 1/.style={sibling distance=3cm},
            level 2/.style={sibling distance=2cm},
            level 3/.style={sibling distance=2cm}
        ]
        \node {$ E $}
        child {
                node {$ E $}
                child {
                        node {$ x $}
                    }
            }
        child {
                node {$ + $}
            }
        child {
                node {$ E $}
                child {
                        node {$ y $}
                    }
                child {
                        node {$ * $}
                    }
                child {
                        node {$ z $}
                    }
            };
    \end{tikzpicture}
\end{figure}

\begin{figure}[H]
    \centering
    \begin{tikzpicture}[
            -,
            level distance=1.5cm,
            level 1/.style={sibling distance=3cm},
            level 2/.style={sibling distance=2cm},
            level 3/.style={sibling distance=2cm}
        ]
        \node {$ E $}
        child {
                node {$ E $}
                child {
                        node {$ x $}
                    }
                child {
                        node {$ + $}
                    }
                child {
                        node {$ y $}
                    }
            }
        child {
                node {$ * $}
            }
        child {
                node {$ E $}
                child {
                        node {$ z $}
                    }
            };
    \end{tikzpicture}
\end{figure}

\vspace{0.5cm}

产生二义性的原因在于运算符之间的优先级在文法中并没有体现。消除二义性的办法就是在文法中引入一个中间量。

\vspace{-1cm}

\begin{align*}
    E & \rightarrow E + T\ |\ T \\
    T & \rightarrow T * F\ |\ F \\
    F & \rightarrow id\ |\ (E)
\end{align*}

\begin{figure}[H]
    \centering
    \begin{tikzpicture}[
            -,
            level distance=1.5cm,
            level 1/.style={sibling distance=3cm},
            level 2/.style={sibling distance=2cm},
            level 3/.style={sibling distance=2cm}
        ]
        \node {$ E $}
        child {
                node {$ E $}
                child {
                        node {$ T $}
                        child {
                                node {$ F $}
                                child {
                                        node {$ x $}
                                    }
                            }
                    }
            }
        child {
                node {$ + $}
            }
        child {
                node {$ T $}
                child {
                        node {$ T $}
                        child {
                                node {$ F $}
                                child {
                                        node {$ y $}
                                    }
                            }
                    }
                child {
                        node {$ * $}
                    }
                child {
                        node {$ F $}
                        child {
                                node {$ z $}
                            }
                    }
            };
    \end{tikzpicture}
\end{figure}

\newpage

\section{CNF}

\subsection{上下文无关语言}

CFG可以用来表示语言$ \{a^nb^n\ |\ n > 0\} $:

\vspace{-1cm}

\begin{align*}
    S & \rightarrow aSb \\
    S & \rightarrow ab
\end{align*}

例如根据$ S $可以生成生成aaabbb:

\vspace{-1cm}

\begin{align*}
    S & \Rightarrow aSb    \\
      & \Rightarrow aaSbb  \\
      & \Rightarrow aaabbb
\end{align*}

CFG好还可以用于表示$ a $和$ b $出现相等次数的语言,例如babaab:

\vspace{-1cm}

\begin{align*}
    S & \rightarrow aB\ |\ bA        \\
    A & \rightarrow a\ |\ aS\ |\ bAA \\
    B & \rightarrow b\ |\ bS\ |\ aBB
\end{align*}

设计CFG需要一定的创造力,大部分复杂的CFG可以由多个简单的CFG并集组成。\\

例如设计一个能够表示语言$ \{0^n1^n\ |\ n \ge 0\} \cup \{1^n0^n\ |\ n \ge 0\} $的CFG。\\

这两个部分可以分别表示为:

\vspace{-1cm}

\begin{align*}
    S_1 & \rightarrow 0S_{1}1\ |\ \epsilon \\
    S_2 & \rightarrow 1S_{2}0\ |\ \epsilon
\end{align*}

只需合并这两个部分,即可得到最终的CFG:

\vspace{-1cm}

\begin{align*}
    S   & \rightarrow S_1\ |\ S_2          \\
    S_1 & \rightarrow 0S_{1}1\ |\ \epsilon \\
    S_2 & \rightarrow 1S_{2}0\ |\ \epsilon
\end{align*}

\vspace{0.5cm}

\subsection{乔姆斯基范式(CNF, Chomsky Normal Form)}

CFG可以被转换为一些更加简洁的形式,如CNF。\\

CNF规定每条CFG的每一条规则都必须满足:

\begin{enumerate}
    \item $ A \rightarrow BC $
    \item $ A \rightarrow a $
\end{enumerate}

其中

\newpage