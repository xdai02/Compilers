\chapter{代码生成}

\section{类型检查}

\subsection{类型检查(Type Checking)}

经过词法分析后,词法分析器将token流传递给语法分析器,语法分析器将生成一个语法树。当源代码转换为语法树时,类型检查器起着至关重要的作用。通过查看语法树,您可以判断每种数据类型是否正在处理正确的变量。\\

类型检查主要是为了判断变量或者参数的实际类型和声明的类型是否匹配。类型检查可以及早地发现类型不匹配的问题,不要等到执行的时候才发现问题。提前发现问题可以降低成本。\\

每种语言都有自己的语言类型规则集,编译器必须检查源程序是否遵循语言的句法和语义约定。它限制程序员在某些情况下可以使用的类型并将类型分配给值。\\

编译器需要检查对象的类型并在违反的情况下报告类型错误,并纠正不正确的类型。\\

如果要由编译器自动完成从一种类型到另一种类型的转换,则称为隐式转换(implicit conversion)。例如整数可以转换为实数,而实数不能转换为整数。而需要由程序员明确指定的类型转换被称为显示类型转换(explicit conversion),也称为强制类型转换。\\

\subsection{静态类型检查(Staic Type Checking)}

静态类型检查发生在编译期间,它在编译时检查变量的类型,这意味着变量的类型在编译时是已知的。\\

静态类型检查包括:

\begin{itemize}
    \item 类型检查:如果将运算符应用于不匹配的操作数,编译器需要报告错误。例如将一个数组变量和函数相加。

    \item 控制流检查:一些能够导致离开某个控制结构的语句,应该存在能够被转移到的位置。例如C中的break语句会导致离开离它最近的while、for或switch结构,如果这样的结构不存在,则会发生错误。

    \item 唯一性检查:在某些情况下,对象只可以定义一次。例如Pascal中标识符必须唯一声明,case语句中的标签必须是不同的。

    \item 名称检查:有时同一个名称可能会出现两次或多次。例如Ada中循环的名称可能出现在结构的开头和结尾。编译器必须检查两个地方是否使用了相同的名称。
\end{itemize}

静态类型检查能够有效发现语法错误和错误的名称,例如额外的标点符号或写错了预定义的名称。对于函数而言,可以检查函数参数的类型和数量是否匹配。\\

\subsection{动态类型检查(Dynamic Type Checking)}

动态类型检查发生在运行时,即当涉及到具体的数据值时才进行类型检查。动态类型检查提供了更宽松、灵活的程序设计环境,在交互式语言中十分有用。\\

动态类型语言一般是脚本语言,如Perl、Ruby、Python、PHP、JavaScript等,可以更快地编写代码,不必每次都指定类型。\\

但是动态类型检查有着一些缺陷。首先它增加了程序的运行时间,影响了效率。同时动态类型检查错误发现太晚,不能防止运行时产生出错。

\newpage

\section{运行时环境}

\subsection{运行时环境(Runtime Environment)}

